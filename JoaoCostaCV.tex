%%%%%%%%%%%%%%%%%%%%%%%%%%%%%%%%%%%%%%%%%
% Twenty Seconds Resume/CV
% LaTeX Template
%
% Original author:
% Carmine Spagnuolo (cspagnuolo@unisa.it) with major modifications by
% Vel (vel@LaTeXTemplates.com) and than modified by
% JoaoCostaIFG (joaocosta.work@posteo.net)
%
% License:
% The MIT License (see included LICENSE file)
%
%%%%%%%%%%%%%%%%%%%%%%%%%%%%%%%%%%%%%%%%%

%----------------------------------------------------------------------------------------
%	PACKAGES AND OTHER DOCUMENT CONFIGURATIONS
%----------------------------------------------------------------------------------------

\documentclass[letterpaper]{twentysecondcv} % a4paper for A4

%----------------------------------------------------------------------------------------
%	 PERSONAL INFORMATION
%----------------------------------------------------------------------------------------

% If you don't need one or more of the below, just remove the content leaving the command, e.g. \cvnumberphone{}

\profilepic{me.png} % Profile picture

\cvname{João de Jesus Costa} % Your name
\cvjobtitle{Software Engineer} % Job title/career

%\cvdate{} % Date of birth
\cvaddress{Santa Maria da Feira, Portugal} % Short address/location, use \newline if more than 1 line is required
\cvnumberphone{+351 912768196} % Phone number
\cvsite{joaocosta.dev} % Personal website
\cvmail{jobs@joaocosta.dev} % Email address
\cvlinkedin{www.linkedin.com/in/joaocostaifg} % LinkedIn
\cvgithub{github.com/JoaoCostaIFG} % GitHub

%----------------------------------------------------------------------------------------

\begin{document}

%----------------------------------------------------------------------------------------
%	 ABOUT ME
%----------------------------------------------------------------------------------------

\aboutme{I'm a software engineer with a Master's in Informatics Engineering.
I like to develop personal software projects, take care of my servers, and write on my
blog.}

%----------------------------------------------------------------------------------------
%	 SKILLS
%----------------------------------------------------------------------------------------

% Skill bar section, each skill must have a value between 0 an 6 (float)
\skills{
  {JavaScript - TypeScript/4},
  {Java/4},
  {Docker/4.5},
  {Python/5},
  {Linux/5},
  {C - C++/5}}

% Skill text section, each skill must have a value between 0 an 6
%\skillstext{}

\languageskills{
  {Portuguese/Native},
  {English/C1}}

\softskills{problem-solving, flexibility, communication}

%----------------------------------------------------------------------------------------

\makeprofile % Print the sidebar

\vfill

%----------------------------------------------------------------------------------------
%	 Bio
%----------------------------------------------------------------------------------------

\section{Bio}

I like working in almost everything software related. I have a somewhat odd mix of work
experiences, but my longest held positions mainly focused on embedded software
development and safety-critical systems.

I get most of my knowledge/experience from personal projects outside of work. These
projects involve informatics and electronics, spanning the whole tech stack. I'm also
a big fan of open-source software and try to contribute to it as much as I can.

In my free time, you can find me working on personal projects, reading, or playing
video games. I'm also a big fan of paintball and cycling.

\vfill

%----------------------------------------------------------------------------------------
%	 EDUCATION
%----------------------------------------------------------------------------------------

\section{Education}

\begin{twenty}
	%\twentyitem{<dates>}{<title>}{<location>}{<description>}
	\twentyitem{2021 - 2023}{Master in Informatics Engineering}
    {Faculty of Engineering of the University of Porto (FEUP)}
    {\emph{Final grade 18/20}}
	\twentyitem{2018 - 2021}{Bachelor of Science}
    {Faculty of Engineering of the University of Porto (FEUP)}
    {\emph{Final grade 18/20}}
\end{twenty}

\vfill

%----------------------------------------------------------------------------------------
%	 WORK EXPERIENCE
%----------------------------------------------------------------------------------------

\section{Work experience}

\begin{twenty}
	%\twentyitem{<dates>}{<title>}{<company>}{<description>}
	\twentyitem{2024→present}{Software Engineer}{Synopsys}
    {Software Engineer developing tools and CI/CD pipelines to support
    the development of drivers for baremetal systems.\\
    \textbf{Tech}: C, Python, Bash}
	\twentyitem{2024}{Backend Software Engineer}{Codacy}
    {Backend Software Engineer working on a new security-focused product of
    the company. Worked mostly with Scala code-bases. Tasks included
    migrating parts of old backend components, implementing new endpoints, and
    refactoring databases. Also did some frontend work in React.\\
    Short stay due to the company's financial problems and mass layoffs.\\
    \textbf{Tech}: Scala, Java, React}
	\twentyitem{2023→2024}{Software Engineer}{Civil Engineering Department of FEUP}
    {Solo freelancer work developing monitoring tools and pipelines for
    construction work that was taking place in the metro tunnels of the city of
    Porto. The objective was to acquire and process real-time sensor data and
    guarantee the safety of the workers and the historical buildings in the area.\\
    Worked part-time, mostly during weekends, while working at Critical Software.\\
    \textbf{Tech}: Python, C\#}
	\twentyitem{2022→2024}{Embedded Software Engineer}{Critical Software}
    {Embedded Software Engineer working on the verification and validation of
    operating systems for the aerospace industry. Responsible for writing
    requirements, test cases, and test procedures on a team of 20 people.\\
    Started as an intern working on my master's thesis.\\
    \textbf{Tech}: C, Python, Arm ASM}
	\twentyitem{2020→2021}{Web Game Developer}{LusoInfo}
    {Member of a 4 people freelancer team tasked with porting a collection
    of flash games to the Unity game engine. The games are part of an educational
    game suite for school children.\\
    Worked part-time, after school hours, during university.\\
    \textbf{Tech}: Unity, C\#}
\end{twenty}

\vfill

%----------------------------------------------------------------------------------------
%	 SECOND PAGE
%----------------------------------------------------------------------------------------

\newpage % Start a new page

\makeprofile % Print the sidebar

% \vfill

%----------------------------------------------------------------------------------------
%	 AWARDS
%----------------------------------------------------------------------------------------

\section{Awards}

\begin{twentyshort}
  %\twentyitemshort{<dates>}{<title/description>}
  \twentyitemshort{2023}{2nd place on AI for banking competition from Millennium}
  \\
  \twentyitemshort{2018}{University scholarship from Amorim Cork Composites}
  \\
  \twentyitemshort{2018}{Highest grade in informatics from highschool}
\end{twentyshort}

\vfill


%----------------------------------------------------------------------------------------
%	 Projects
%----------------------------------------------------------------------------------------

\section{Projects}

\begin{twenty}
  \twentyitem{2023}{Octree implementation for SLAM}
    {\twentyshortghref{https://github.com/JoaoCostaIFG/slam}{Project's GitHub page}}
    {An Octree implementation in C++ for use in SLAM application for an autonomous
    submarine. This project was developed for the university's robotics research
    institute. The objective was improving the performance of existing Octree
    implementations for the current application without compromising the
    spatial efficiency and accuracy.}
  \twentyitem{2022}{TraSMAPy}
    {\twentyshortghref{https://github.com/JoaoCostaIFG/TraSMAPy}{Project's GitHub page}}
    {A Python API for the \href{https://www.eclipse.org/sumo/}{SUMO traffic simulator} with the
    objective of allowing researchers with limited knowledge in informatics and programming to
    model road traffic simulations. The API abstracts lower level concepts of the simulator and
    builds new concepts to make it easier to build agent-based simulations,}
  \twentyitem{2022}{µKernel}
    {\twentyshortghref{https://github.com/JoaoCostaIFG/SETR/tree/master/uKernel}{Project's GitHub page}}
    {A real-time micro-kernel for the Arduino UNO supporting POSIX style tasks, mutexes,
    EDF scheduling, task sleeping, and assertions. Responsible for most of the implementation
    and research for the project.}
  \twentyitem{2021}{Dry Beans classification}
    {\twentyshortghref{https://github.com/JoaoCostaIFG/IART/tree/master/Proj2}{Project's GitHub page}}
    {Model (unsupervised learning) to classify beans between seven classes with similar
    features. This served as an opportunity to gain experience in data science and machine
    learning libraries for python. Responsible for part of the data analysis and treatment,
    the application of unsupervised learning algorithms, and the comparison of the results.}
  \twentyitem{2021}{Segmentation Fault}
    {\twentyshortghref{https://github.com/JoaoCostaIFG/LBAW/wiki/pa}{Project's GitHub page}}
    {Online QA web application resembling \twentyshortref{https://stackoverflow.com}{StackOverflow}.
    This project was developed with the purpose of learning Bootstrap, Docker, Laravel,
    and PostgreSQL. Responsible for the Docker image, database design and indexes,
    most of the front-end, and accessibility/design choices.}
  \twentyitem{2020-2021}{MAWW}
    {\twentyshortghref{https://github.com/JoaoCostaIFG/MAWW}{Project's GitHub page}}
    {Program that allows users (using X11) on Linux to have animated backgrounds,
    regardless of their compositor. This project is available as an
    \twentyshortref{https://aur.archlinux.org/packages/maww}{Arch Linux package}
    on the AUR.}
\end{twenty}

\vfill

%----------------------------------------------------------------------------------------
%	 PUBLICATIONS
%----------------------------------------------------------------------------------------

\section{Publications}

\begin{twenty}
	%\twentyitem{<dates>}{<title>}{<location>}{<description>}
	\twentyitem{2023}{Verification of real-time operating system for DO-178C compliance}
    {}
    {My Master's thesis. The objective of my thesis was to analyze and optimize
    the processes of the aviation real-time OS verification team at Critical
    Software. This work resulted in several internal tools and optimizations
    that streamlined the development and saved money.}
\end{twenty}

%----------------------------------------------------------------------------------------

\vfill

\end{document}
