%%%%%%%%%%%%%%%%%%%%%%%%%%%%%%%%%%%%%%%%%
% Twenty Seconds Resume/CV
% LaTeX Template
%
% Original author:
% Carmine Spagnuolo (cspagnuolo@unisa.it) with major modifications by
% Vel (vel@LaTeXTemplates.com) and than modified by
% JoaoCostaIFG (joaocosta.work@posteo.net)
%
% License:
% The MIT License (see included LICENSE file)
%
%%%%%%%%%%%%%%%%%%%%%%%%%%%%%%%%%%%%%%%%%

%----------------------------------------------------------------------------------------
%	PACKAGES AND OTHER DOCUMENT CONFIGURATIONS
%----------------------------------------------------------------------------------------

\documentclass[letterpaper]{twentysecondcv} % a4paper for A4

%----------------------------------------------------------------------------------------
%	 PERSONAL INFORMATION
%----------------------------------------------------------------------------------------

% If you don't need one or more of the below, just remove the content leaving the command, e.g. \cvnumberphone{}

\profilepic{me.png} % Profile picture

\cvname{João de Jesus Costa} % Your name
\cvjobtitle{Software Engineer} % Job title/career

%\cvdate{} % Date of birth
\cvaddress{Santa Maria da Feira, Portugal} % Short address/location, use \newline if more than 1 line is required
\cvnumberphone{+351 912768196} % Phone number
\cvsite{joaocosta.dev} % Personal website
\cvmail{jobs@joaocosta.dev} % Email address
\cvlinkedin{www.linkedin.com/in/joaocostaifg} % LinkedIn
\cvgithub{github.com/JoaoCostaIFG} % GitHub

%----------------------------------------------------------------------------------------

\begin{document}

%----------------------------------------------------------------------------------------
%	 ABOUT ME
%----------------------------------------------------------------------------------------

\aboutme{I'm a software engineer with a Master's in Informatics Engineering.
I like to develop personal software projects, take care of my servers, and write on my
blog.}

%----------------------------------------------------------------------------------------
%	 SKILLS
%----------------------------------------------------------------------------------------

% Skill bar section, each skill must have a value between 0 an 6 (float)
\skills{
  {JavaScript - TypeScript/4},
  {Java/4.5},
  {Docker/4.5},
  {Python/5},
  {Linux/5},
  {C - C++/5.5}}

% Skill text section, each skill must have a value between 0 an 6
%\skillstext{}

\languageskills{
  {Portuguese/Native},
  {English/C1}}

\softskills{problem-solving, flexibility, communication}

%----------------------------------------------------------------------------------------

\makeprofile % Print the sidebar

\vfill

%----------------------------------------------------------------------------------------
%	 Bio
%----------------------------------------------------------------------------------------

\section{Bio}

I like working in almost everything software related. My work experience mainly focuses
on embedded software development, focusing on safety-critical systems, but I consider
myself a generalist. This last year I started working in backend software development
at a startup.

I get most of my knowledge/experience from my personal projects outside of work. These
projects involve informatics and electronics, spanning the whole tech stack. I'm also
a big fan of open-source software and try to contribute to it as much as I can.

In my free time, you can find me working on personal projects, reading, or playing
video games. I'm also a big fan of paintball and cycling.

\vfill

%----------------------------------------------------------------------------------------
%	 AWARDS
%----------------------------------------------------------------------------------------

\section{Awards}

\begin{twentyshort}
  %\twentyitemshort{<dates>}{<title/description>}
  \twentyitemshort{2023}{2nd place on AI for banking competition from Millenium}
  \\
  \twentyitemshort{2018}{University scholarship from Amorim Cork Composites}
  \\
  \twentyitemshort{2018}{Highest grade in informatics from highschool}
\end{twentyshort}

\vfill

%----------------------------------------------------------------------------------------
%	 EDUCATION
%----------------------------------------------------------------------------------------

\section{Education}

\begin{twenty}
	%\twentyitem{<dates>}{<title>}{<location>}{<description>}
	\twentyitem{2021 - 2023}{Master in Informatics Engineering}
    {Faculty of Engineering of the University of Porto (FEUP)}
    {\emph{Final grade 18/20}}
	\twentyitem{2018 - 2021}{Bachelor of Science}
    {Faculty of Engineering of the University of Porto (FEUP)}
    {\emph{Final grade 18/20}}
\end{twenty}

\vfill

%----------------------------------------------------------------------------------------
%	 WORK EXPERIENCE
%----------------------------------------------------------------------------------------

\section{Work experience}

\begin{twenty}
	%\twentyitem{<dates>}{<title>}{<company>}{<description>}
	\twentyitem{2024}{Backend Software Engineer}{Codacy}
    {Backend Software Engineer on the team responsible for the new security-focused
    product of the company. Worked mostly with Scala code-bases. Tasks included
    migrating parts of old backend components, implementing new endpoints, and
    refactoring databases.}
	\twentyitem{2022-2024}{Embedded Software Engineer}{Critical Software}
    {Embedded Software Engineer working on the verification and validation of
    operating systems for the aerospace industry. Responsible for writing
    requirements, test cases, and test procedures for the verification of
    real-time operating systems on a team of 20 people.}
	\twentyitem{2023-2024}{Software Engineer}{Civil Engineering Department of FEUP}
    {Solo freelancer work with the civil engineering department of FEUP to develop
    monitoring tools and pipelines for the construction work that was taking
    place in the metro tunnels of the city of Porto, Portugal. The objective
    was to acquire and process real-time sensor data and guarantee the safety
    of the workers, equipment, and historical buildings in the area.}
	\twentyitem{2022}{Software Engineer Intern}{Critical Software}
    {Internship program at Critical Software, with the objective of learning
    more about the company's activities, agile methodologies, and developing
    soft-skills.}
	\twentyitem{2020-2021}{Web Game Developer}{LusoInfo}
    {Member of a 4 people freelancer team tasked with porting a collection
    of flash games to the Unity game engine. The games are part of an educational
    game suite for school children}
\end{twenty}

\vfill

%----------------------------------------------------------------------------------------
%	 SECOND PAGE EXAMPLE
%----------------------------------------------------------------------------------------

\newpage % Start a new page

\makeprofile % Print the sidebar

% \vfill

%----------------------------------------------------------------------------------------
%	 Projects
%----------------------------------------------------------------------------------------

\section{Projects}

\begin{twenty}
  \twentyitem{2023}{Octree implementation for SLAM}
    {\twentyshortghref{https://github.com/JoaoCostaIFG/slam}{Project's GitHub page}}
    {An Octree implementation in C++ for use in SLAM application for an autonomous
    submarine. This project was developed for the university's robotics research
    institute. The objective was improving the performance of existing Octree
    implementations for the current application without compromising the
    spacial efficiency and accuracy.}
  \twentyitem{2022}{TraSMAPy}
    {\twentyshortghref{https://github.com/JoaoCostaIFG/TraSMAPy}{Project's GitHub page}}
    {A Python API for the \href{https://www.eclipse.org/sumo/}{SUMO traffic simulator} with the
    objective of allowing researchers with limited knowledge in informatics and programming to
    model road traffic simulations. The API abstracts lower level concepts of the simulator and
    builds new concepts to make it easier to build agent-based simulations,}
  \twentyitem{2022}{µKernel}
    {\twentyshortghref{https://github.com/JoaoCostaIFG/SETR/tree/master/uKernel}{Project's GitHub page}}
    {A real-time micro-kernel for the Arduino UNO supporting POSIX style tasks, mutexes,
    EDF scheduling, task sleeping, and assertions. Responsible for most of the implementation
    and research for the project.}
  \twentyitem{2021}{Dry Beans classification}
    {\twentyshortghref{https://github.com/JoaoCostaIFG/IART/tree/master/Proj2}{Project's GitHub page}}
    {Model (unsupervised learning) to classify beans between seven classes with similar
    features. This served as an opportunity to gain experience in data science and machine
    learning libraries for python. Responsible for part of the data analysis and treatment,
    the application of unsupervised learning algorithms, and the comparison of the results.}
  \twentyitem{2021}{Segmentation Fault}
    {\twentyshortghref{https://github.com/JoaoCostaIFG/LBAW/wiki/pa}{Project's GitHub page}}
    {Online QA web application resembling \twentyshortref{https://stackoverflow.com}{StackOverflow}.
    This project was developed with the purpose of learning Bootstrap, Docker, Laravel,
    and PostgreSQL. Responsible for the Docker image, database design and indexes,
    most of the front-end, and accessibility/design choices.}
  \twentyitem{2020-2021}{MAWW}
    {\twentyshortghref{https://github.com/JoaoCostaIFG/MAWW}{Project's GitHub page}}
    {Program that allows users (using X11) on Linux to have animated backgrounds,
    regardless of their compositor. This project is available as an
    \twentyshortref{https://aur.archlinux.org/packages/maww}{Arch Linux package}
    on the AUR.}
\end{twenty}

\vfill

%----------------------------------------------------------------------------------------
%	 PUBLICATIONS
%----------------------------------------------------------------------------------------

\section{Publications}

\begin{twenty}
	%\twentyitem{<dates>}{<title>}{<location>}{<description>}
	\twentyitem{2023}{Verification of real-time operating system for DO-178C compliance}
    {}
    {My Master's thesis. The objective of my thesis was to analyze and optimize
    the processes of the aviation real-time OS verification team at Critical
    Software. This work resulted in several internal tools and optimizations
    that streamlined the development and saved money.}
\end{twenty}

%----------------------------------------------------------------------------------------

\vfill

\end{document}
