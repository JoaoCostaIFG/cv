%%%%%%%%%%%%%%%%%%%%%%%%%%%%%%%%%%%%%%%%%
% Twenty Seconds Resume/CV
% LaTeX Template
% Version 1.1 (8/1/17)
%
% This template has been downloaded from:
% http://www.LaTeXTemplates.com
%
% Original author:
% Carmine Spagnuolo (cspagnuolo@unisa.it) with major modifications by 
% Vel (vel@LaTeXTemplates.com)
%
% License:
% The MIT License (see included LICENSE file)
%
%%%%%%%%%%%%%%%%%%%%%%%%%%%%%%%%%%%%%%%%%

%----------------------------------------------------------------------------------------
%	PACKAGES AND OTHER DOCUMENT CONFIGURATIONS
%----------------------------------------------------------------------------------------

\documentclass[letterpaper]{twentysecondcv} % a4paper for A4

%----------------------------------------------------------------------------------------
%	 PERSONAL INFORMATION
%----------------------------------------------------------------------------------------

% If you don't need one or more of the below, just remove the content leaving the command, e.g. \cvnumberphone{}

\profilepic{me.png} % Profile picture

\cvname{João de Jesus Costa} % Your name
\cvjobtitle{Software Engineer} % Job title/career

%\cvdate{} % Date of birth
\cvaddress{Santa Maria da Feira, Portugal} % Short address/location, use \newline if more than 1 line is required
\cvnumberphone{(+351) 912768196} % Phone number
\cvsite{https://joaocosta.dev} % Personal website
\cvmail{joaocosta.work@posteo.net} % Email address
\cvgithub{https://github.com/JoaoCostaIFG} % GitHub

%----------------------------------------------------------------------------------------

\begin{document}

%----------------------------------------------------------------------------------------
%	 ABOUT ME
%----------------------------------------------------------------------------------------

\aboutme{
  Software engineer currently enrolled in a Master's degree in Informatics and Computing
  Engineering at FEUP. Interests mainly reside in developing software, server
  administration, writing, and electronics. Has experience conducting workshops
  related to informatics as a member of IEEE's University of Porto student branch, and
  working in large teams.
}

%----------------------------------------------------------------------------------------
%	 SKILLS
%----------------------------------------------------------------------------------------

% Skill bar section, each skill must have a value between 0 an 6 (float)
\skills{
  {Python/4},
  {Linux/4},
  {Java/4},
  {DevOps/3},
  {C - C++/4.5}}

% Skill text section, each skill must have a value between 0 an 6
\skillstext{}

\languageskills{
  {Portuguese/Native},
  {English/C1}}

\softskills{communication, problem-solving, flexibility}

%----------------------------------------------------------------------------------------

\makeprofile % Print the sidebar

\vspace*{\fill} 

%----------------------------------------------------------------------------------------
%	 HOBBIES / INTERESTS
%----------------------------------------------------------------------------------------

\section{Hobbies/Interests}

Working on personal projects involving informatics and (sometimes) electronics. Often,
these projects lead to contributions to open-source software. Enjoys maintaining a
personal server and blog.

%----------------------------------------------------------------------------------------
%	 EDUCATION
%----------------------------------------------------------------------------------------

\section{Education}

\begin{twenty}
	%\twentyitem{<dates>}{<title>}{<location>}{<description>}
	\twentyitem{2021 - Current}{Master of Science}
    {Faculty of Engineering of the University of Porto (FEUP)}
    {\emph{Majoring in Informatics Engineering}}
	\twentyitem{2018 - 2021}{Bachelor of Science}
    {Faculty of Engineering of the University of Porto (FEUP)}
    {\emph{Final grade 18/20}}
\end{twenty}

%----------------------------------------------------------------------------------------
%	 PUBLICATIONS
%----------------------------------------------------------------------------------------

%\section{Publications}
%
%\begin{twentyshort}
%	%\twentyitemshort{<dates>}{<title/description>}
%	\twentyitemshort{1865}{Chapter One, Down the Rabbit Hole.}
%	\twentyitemshort{1865}{Chapter Two, The Pool of Tears.}
%\end{twentyshort}

%----------------------------------------------------------------------------------------
%	 AWARDS
%----------------------------------------------------------------------------------------

%\section{Awards}
%
%\begin{twentyshort}
%	%\twentyitemshort{<dates>}{<title/description>}
%	\twentyitemshort{1987}{All-Time Best Fantasy Novel.}
%\end{twentyshort}

%----------------------------------------------------------------------------------------
%	 WORK EXPERIENCE
%----------------------------------------------------------------------------------------

\section{Work experience}

\begin{twenty}
	%\twentyitem{<dates>}{<title>}{<company>}{<description>}
	\twentyitem{2022}{Software developer}{Critical Software}
    {Enrolled in the summer internship program at Critical Software (Summer Camp), with
    the objective of learning more about the company's activities, and developing
    soft-skills. Developed three projects, and attended multiple workshops during
    this one-month internship.}
	\twentyitem{2020-2021}{Computer games developer}{LusoInfo}
    {Responsible from the team tasked to recreate a set of educational games, developed
    in flash, in the Unity game engine. These games were then hosted in the company's
    website to be played by children (aged 6 to 14) at school. Developed the games
    in C\# for the Unity game engine, with an emphasis on performance so the games
    ran on WebGL on lower-end computers.}
\end{twenty}

%----------------------------------------------------------------------------------------
%	 Projects
%----------------------------------------------------------------------------------------

\section{Projects}

\begin{twenty}
	%\twentyitem{<dates>}{<title>}{<company>}{<description>}
  \twentyitem{2022}{µKernel}
    {\twentyshortghref{https://github.com/JoaoCostaIFG/SETR/tree/master/uKernel}{Project's GitHub page}}
    {A real-time micro-kernel for the Arduino UNO supporting POSIX style tasks, mutexes,
    EDF scheduling, task sleeping, and assertions. Responsible for most of the implementation
    and research for the project.}
  \twentyitem{2021}{Dry Beans classification}
    {\twentyshortghref{https://github.com/JoaoCostaIFG/IART/tree/master/Proj2}{Project's GitHub page}}
    {Model (unsupervised learning) to classify beans between seven classes with similar
    features. This served as an opportunity to gain experience in data science and machine
    learning libraries for python. Responsible for part of the data analysis and treatment,
    the application of unsupervised learning algorithms, and the comparison of the results.}
  \twentyitem{2021}{Segmentation Fault}
    {\twentyshortghref{https://github.com/JoaoCostaIFG/LBAW/wiki/pa}{Project's GitHub page}}
    {Online QA web application resembling \twentyshortref{https://stackoverflow.com}{StackOverflow}.
    This project was developed with the purpose of learning Bootstrap, Docker, Laravel,
    and PostgreSQL. Responsible for the Docker image, database design and indexes,
    most of the front-end, and accessibility/design choices.}
  \twentyitem{2020-2021}{MAWW}
    {\twentyshortghref{https://github.com/JoaoCostaIFG/MAWW}{Project's GitHub page}}
    {Program that allows users (using X11) on Linux to have animated backgrounds,
    regardless of their compositor. This project is available as an
    \twentyshortref{https://aur.archlinux.org/packages/maww}{Arch Linux package}
    on the AUR.}
\end{twenty}

%----------------------------------------------------------------------------------------
%	 SECOND PAGE EXAMPLE
%----------------------------------------------------------------------------------------

%\newpage % Start a new page

%\makeprofile % Print the sidebar

%\section{Other information}

%\subsection{Review}

%TEXT

%----------------------------------------------------------------------------------------

\vfill

\end{document} 
